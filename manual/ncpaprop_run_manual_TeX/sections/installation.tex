\section{Installation notes}

\subsection{Prerequisites}

\subsubsection{Operating System}

\textbf{ncpaprop} is designed to run in a UNIX environment. It has been tested on Ubuntu and Fedora Linux operating systems, and MacOS Mountain Lion and Yosemite (10.8-10.10). 

\subsubsection{Libraries}

Fourier transforms are performed in \textbf{ncpaprop} using the \textbf{FFTW} libraries, http://www.fftw.org. The \textbf{GSL} libraries, http://www.gnu.org/software/gsl/, are used for interpolation and splining. The \textbf{SLEPc} library, http://www.slepc.upv.es/, is used to solve the large dimensional eigenvalue problems required for the normal mode models. \textbf{SLEPc} requires the \textbf{PETSc} libraries, http://www.mcs.anl.gov/petsc/, to which it can be seen as a submodule. The current version of \textbf{ncpaprop} is designed to use, and has been tested with, versions 3.2-p3 of \textbf{SLEPc} and 3.2-p5 of \textbf{PETSc}. As these are not the current versions of \textbf{SLEPc} and \textbf{PETSc} they are packaged with the \textbf{ncpaprop} distribution. 

\subsubsection{Compilers}

\textbf{ncpaprop} is written in C++ and was built and tested using the GNU C++ compiler \textbf{G++}.  The \textbf{PETSc} and \textbf{SLEPc} libraries are partially written in Fortran and were built and tested using the GNU Fortran compiler \textbf{gfortran}.  Both of these are required, and they must be of the same version.  

\subsection{Procedure}

\textbf{ncpaprop} uses a standard GNU-style configure script to detect and locate the above prerequisites and generate appropriate makefiles.  The \texttt{configure} script will search the standard library locations, plus if Fink is detected it will search /sw/lib. An additional flag can be invoked to allow the installer to automatically download and install the prerequisites in system-standard locations if desired. The details of this process are explained below.

\subsubsection{Installation Overview}

To begin installation, uncompress the distribution with 

\texttt{tar xvzf ncpaprop-{\version}.tar.gz}

\noindent
and cd into \texttt{ncpaprop-\version}.  Then run

\texttt{./configure}

\noindent
to have the installer search for prerequisites and exit with an explanatory message if one is missing.  Optional flags that can be used by the \texttt{configure} script include:

\begin{description}
\item[\texttt{--enable-auto-dependencies}]\hfill \\
\noindent
Attempt to download and install missing prerequisites automatically. Use this option only if installation in the default installation locations is satisfactory. The \textbf{ncpaprop} installer supports the \textbf{apt-get} and \textbf{yum} package managers on Linux, and \textbf{port} and \textbf{fink} on MacOS.  The exact commands run by each package manager are detailed in later sections.

\item[\texttt{--disable-library-guess}]\hfill \\
\noindent
If the detected MacOS package manager is Fink, the standard C++ libraries are often not placed in a default-searched location.  To handle this, the installer will attempt to guess the location of the libraries and add that guess to the linker's search path.  Use this flag to disable this behavior.

\item[\texttt{LDFLAGS="-L/path/to/library"}]\hfill \\
\noindent
The user can also specify additional library search paths using this syntax, often in conjunction with \texttt{--disable-library-guess}. For example if the \textbf{GSL} or \textbf{FFTW} libraries are placed in a non-standard location this option can be used to provide the correct path. 
\end{description}

\noindent The following tests are run by the \texttt{configure} script:

\subsubsection{Shells and Interpreters}

The configuration script begins by checking for the existence of \texttt{bash} and \texttt{perl} executables.  If these are not both found, the installer exits with an error message.  This should never happen on a properly functioning system, but is included as a sanity check.

\subsubsection{Package Managers}

Depending on whether the host OS is Linux or MacOS, the installer searches for a known package manager.  On Linux, the installer will search for \texttt{apt-get} or \texttt{yum}, in that order, while on MacOS the installer will search for \texttt{port} or \texttt{fink}, in that order.  If no package manager is found, and the \texttt{--enable-auto-dependencies} flag is set, the installer will exit with an error message.

\subsubsection{g++}

The installer will search for either \texttt{g++-4} or \texttt{g++} on its path, using the first one it finds.  (\texttt{g++-4} is preferred because its presence indicates that Fink is in use, and Fink does not override the default XCode version of \texttt{g++}).  If neither of these is found, and the \texttt{--enable-auto-dependencies} flag is set, the following commands will be run using \texttt{sudo}:

\begin{table}[h]
\begin{tabular}{r|l}
\textbf{Manager} & \textbf{Command(s)} \\ 
\hline
\textbf{apt-get} & \texttt{apt-get -y install g++} \\
\rowcolor[HTML]{C0C0C0} 
\textbf{yum} & \texttt{yum install gcc-c++} \\
\textbf{port} & \begin{tabular}[c]{@{}l@{}}\texttt{port install gcc48}\\ \texttt{port select --set gcc mp-gcc48}\end{tabular} \\
\rowcolor[HTML]{C0C0C0} 
\textbf{fink} & \texttt{fink install gcc48}
\end{tabular}
\end{table}

MacPorts users should note that the 'port select' command will create symlinks in /opt/local/bin that will take precedence over previously installed versions of gcc, g++, and gfortran.  This can be undone, if desired, using:

\begin{verbatim}
sudo port select --set gcc none
hash -r
\end{verbatim}

Depending on your system setup, you may have to manually add /opt/local/lib/gcc48 to the library search path using \verb%./configure LDFLAGS=-L/opt/local/lib/gcc48%.  The installer will attempt to do this automatically for MacPorts users.

\subsubsection{gfortran}

The installer will search for \texttt{gfortran} on its path.  If this is not found, and the \texttt{--enable-auto-dependencies} flag is set, the following commands will be run using \texttt{sudo}:

\begin{table}[h]
\begin{tabular}{r|l}
\textbf{Manager} & \textbf{Command(s)} \\ 
\hline
\textbf{apt-get} & \texttt{apt-get -y install gfortran} \\
\rowcolor[HTML]{C0C0C0} 
\textbf{yum} & \texttt{yum install gcc-gfortran} \\
\textbf{port} & \begin{tabular}[c]{@{}l@{}}\texttt{port install gcc48}\\ \texttt{port select --set gcc mp-gcc48}\end{tabular} \\
\rowcolor[HTML]{C0C0C0} 
\textbf{fink} & \texttt{fink install gcc48}
\end{tabular}
\end{table}

\noindent
Note that the \texttt{port} and \texttt{fink} commands are identical to those for \texttt{g++}; \texttt{port} and \texttt{fink} install a suite of compilers at once, so installing the package for one compiler should install the others at the same time.

\subsubsection{Compiler Versions}

The installer will then check that the \texttt{g++} and \texttt{gfortran} versions are identical, and issue an error message if they are not.  This possibility is not easily solved automatically, so it is left to the user to determine how it should be solved.

\subsubsection{FFTW}

The installer will then check for the presence of libraries for FFTW version 3.  If they are not found, and the \texttt{--enable-auto-dependencies} flag is set, the following commands are issued on Linux hosts using \texttt{sudo}:

\begin{table}[h]
\begin{tabular}{r|l}
\textbf{Manager} & \textbf{Command(s)} \\ 
\hline
\textbf{apt-get} & \texttt{apt-get -y install libfftw3-dev} \\
\rowcolor[HTML]{C0C0C0} 
\textbf{yum} & \texttt{yum install fftw-devel}
\end{tabular}
\end{table}

\noindent
On MacOS hosts, FFTW is downloaded and installed from source:

\begin{verbatim}
mkdir prereq
cd prereq
curl -O http://www.fftw.org/fftw-3.3.4.tar.gz
tar xvzf fftw-3.3.4.tar.gz
cd fftw-3.3.4
./configure
make
sudo make install
cd ../..
rm -r prereq
\end{verbatim}

\subsubsection{GSL}

The installer will then check for the presence of libraries for the GSL scientific library.  If they are not found, and the \texttt{--enable-auto-dependencies} flag is set, the following commands are issued on Linux hosts using \texttt{sudo}:

\begin{table}[h]
\begin{tabular}{r|l}
\textbf{Manager} & \textbf{Command(s)} \\ 
\hline
\textbf{apt-get} & \texttt{apt-get -y install libgsl0} \\
\rowcolor[HTML]{C0C0C0} 
\textbf{yum} & \texttt{yum install gsl-devel}
\end{tabular}
\end{table}

\noindent
On MacOS hosts, the GSL is downloaded and installed from source:

\begin{verbatim}
mkdir prereq
cd prereq
curl -O ftp://ftp.gnu.org/gnu/gsl/gsl-1.16.tar.gz
tar xvzf gsl-1.16.tar.gz
cd gsl-1.16
./configure --disable-shared --disable-dependency-tracking
make
sudo make install
cd ../..
rm -r prereq
\end{verbatim}

\subsubsection{Compilation}

Once the configuration script has successfully run, the package can be built simply by running

\texttt{make}

\noindent
This will unpack and build the included versions of \textbf{PETSc} and \textbf{SLEPc} locally. This can take some time. It will then compile the libraries and executables that make up \textbf{ncpaprop}.  If the Fink \texttt{g++-4} is the selected compiler, the installer will try to guess the location of the Fink-installed standard C++ libraries (unless the \texttt{--disable-library-guess} flag was set).  The guess uses the reported \texttt{g++-4} version to calculate its path under the Fink path, so for example if Fink is installed in its default \texttt{/sw} location and \texttt{g++-4} is version 4.8.4, the guessed library path will be \texttt{/sw/lib/gcc4.8/lib}.

One common failure mode is for the linker to attempt to use the standard C++ libraries from a different version than the compiler that was used.  This has been more common of a problem for MacOS systems than for Linux systems.  The problem is generally indicated by multiple error messages indicating that various common C++ library functions cannot be found. If this occurs, the user should re-run \texttt{./configure} with the \texttt{--disable-library-guess} flag and, if known, the 

\texttt{LDFLAGS="-L/path/to/C++/libraries"} 

\noindent
instruction.

\subsubsection{Testing}

To check proper functionality, a suite of test scripts have been written to compare the output of the simpler tools with that generated on the development machine.  This allows a sanity check to make sure that identical commands give identical results (within reasonable rounding errors).  To confirm that the compilation has been successful, run

\texttt{make test}

\noindent
and check the results.  If there are errors, check \texttt{test/testlog.txt} for clues as to why things didn't work properly.  A successful test should print the following output to the screen:

\begin{verbatim}
$ make test
make -C test test
make[1]: Entering directory `/home/claus/ncpaprop/v1.3/test'
rm testlog.txt

*** Running 2D Ray Theory Routines ***
/bin/bash run_raytrace.2d_test.bash >> ./testlog.txt

*** Running 3D Ray Theory Routines ***
/bin/bash run_raytrace.3d_test.bash >> ./testlog.txt

*** Running Effective-Sound-Speed Modal Routines ***
/bin/bash run_Modess_test.bash >> ./testlog.txt 

*** Running Complex Effective-Sound-Speed Modal Routines ***
/bin/bash run_CModess_test.bash >> ./testlog.txt 

*** Running One-Way Coupled Modal Routines ***
/bin/bash run_ModessRD1WCM_test.bash >> ./testlog.txt 

*** Running Wide-Angle Modal Routines ***
/bin/bash run_WMod_test.bash >> ./testlog.txt 

*** Checking 2D Ray Theory Calculation Results ***
/usr/bin/perl compare_raytrace.2d_test
raytrace.2d test OK

*** Checking 3D Ray Theory Calculation Results ***
/usr/bin/perl compare_raytrace.3d_test
raytrace.3d test OK

*** Checking Effective-Sound-Speed Modal Calculation Results ***
/bin/bash compare_Modess_test
Column 2 of Modess test OK
Column 3 of Modess test OK
Column 4 of Modess test OK

*** Checking Complex Effective-Sound-Speed Modal Calculation Results ***
/bin/bash compare_CModess_test
Column 2 of CModess test 1 OK
Column 3 of CModess test 1 OK
Column 4 of CModess test 1 OK
Column 5 of CModess test 1 OK

*** Checking One-Way Coupled Modal Calculation Results ***
/bin/bash compare_ModessRD1WCM_test
Column 2 of ModessRD1WCM test OK
Column 3 of ModessRD1WCM test OK
Column 4 of ModessRD1WCM test OK

*** Checking Wide-Angle Modal Calculation Results ***
/bin/bash compare_WMod_test
Column 2 of WMod test OK
Column 3 of WMod test OK
Column 4 of WMod test OK
make[1]: Leaving directory `/home/claus/ncpaprop/v1.3/test'
\end{verbatim}

\subsection{Typesetting this Manual}

This manual was produced with LaTeX. The required \verb+.tex+ and graphics files are archived in the \textbf{ncpaprop} package in the \verb+manual+ directory. The main \verb+.tex+ file, \verb+NCPA_prop_manual.tex+, is contained in the subdirectory \verb+ncpaprop_run_manual_TeX+ alomg with a utitlity script \verb+runtex.sh+. Entering \verb+. runtex.sh+ produces the manual as a pdf file and moves it to the \verb+manual+ directory. Entering \verb+. runtex.sh bib+ produces the bibliography using bibtex. 