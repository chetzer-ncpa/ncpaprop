\section{Installation notes}

\subsection{Prerequisites}

\subsubsection{Operating System}

\textbf{ncpaprop} is designed to run in a UNIX environment. It has been tested on Ubuntu, Red Hat, and CentOS Linux operating systems, and MacOS Mojave (10.14). 

\subsubsection{Libraries}

Fourier transforms are performed in \textbf{ncpaprop} using the \textbf{FFTW} libraries, http://www.fftw.org. The \textbf{GSL} libraries, http://www.gnu.org/software/gsl/, are used for interpolation and splining. The \textbf{SLEPc} library, http://www.slepc.upv.es/, is used to solve the large dimensional eigenvalue problems required for the normal mode models. \textbf{SLEPc} requires the \textbf{PETSc} libraries, http://www.mcs.anl.gov/petsc/, to which it can be seen as a submodule. 

\subsubsection{Compilers}

\textbf{ncpaprop} is written in C++ and was built and tested using the GNU C++ compiler \textbf{g++}, and will also compile under the MacOS native \textbf{clang} compiler.  The \textbf{PETSc} and \textbf{SLEPc} libraries require a corresponding C compiler.

\subsection{Procedure}

\textbf{ncpaprop} uses a standard GNU-style configure script to detect and locate the above prerequisites and generate appropriate makefiles.  The \texttt{configure} script will search the standard library locations. An additional flag can be invoked (for Linux installations only) to allow the installer to automatically download and install the prerequisites using the native package manager. The details of this process are explained below.

\subsubsection{Installation Overview}

To begin installation, select an installation location for the \textbf{ncpaprop} directory to reside.  Clone the GitHub repository's master branch with

\texttt{git clone https://github.com/chetzer-ncpa/ncpaprop.git ncpaprop}

\noindent
and cd into \texttt{ncpaprop}.  Then run

\texttt{./configure}

\noindent
to set up the build environment and make sure all prerequisites are met.  Optional flags that can be used by the \texttt{configure} script include:

\begin{description}
\item[\texttt{--enable-autodependencies}]\hfill \\
\noindent
Attempt to download and install missing prerequisites automatically. Use this option only if installation in the default installation locations is satisfactory. The \textbf{ncpaprop} installer currently supports the \textbf{apt-get} and \textbf{yum} package managers on Linux.  MacOS users should install \textbf{FFTW} and \textbf{GSL} using their method of choice.

\item[\texttt{LDFLAGS="-L/path/to/library"}]\hfill \\
\noindent
The user can also specify additional library search paths using this syntax.
\end{description}

\noindent In addition, the user must specify whether to use an existing \textbf{PETSc}/\textbf{SLEPc} installation (that has had both real and complex architectures built) or to allow the installer to download and build a set of libraries local to the \textbf{ncpaprop} installation.  To link to an existing installation, the user must provide the following, either as environmental variables or as \texttt{key=value} pairs on the \texttt{./configure} command:

\begin{description}
\item[\texttt{PETSC\_DIR}]\hfill \\
\noindent
The root directory of the \textbf{PETSc} installation.

\item[\texttt{SLEPC\_DIR}]\hfill \\
\noindent
The root directory of the \textbf{SLEPc} installation.

\item[\texttt{PETSC\_ARCH\_REAL}]\hfill \\
\noindent
The architecture name selected for the (default) \texttt{--with-scalar-type=real} \textbf{PETSc} build.  This generally defaults to \texttt{arch-\$OS-c-debug} or \texttt{arch-\$OS-c-debug} if the user does not specify, where \texttt{\$OS} represents the operating system.  It will appear as a subdirectory of \texttt{\$PETSC\_DIR} and \texttt{\$SLEPC\_DIR}.

\item[\texttt{PETSC\_ARCH\_COMPLEX}]\hfill \\
\noindent
The architecture name selected for the \texttt{--with-scalar-type=complex} \textbf{PETSc} build.  This generally defaults to \texttt{arch-\$OS-c-complex} or \texttt{arch-\$OS-c-complex} if the user does not specify, where \texttt{\$OS} represents the operating system.  It will appear as a subdirectory of \texttt{\$PETSC\_DIR} and \texttt{\$SLEPC\_DIR}.
\end{description}

\noindent For example:

\noindent \texttt{./configure --enable-autodependencies PETSC\_DIR=/code/petsc SLEPC\_DIR=/code/slepc/slepc-3.12.2 PETSC\_ARCH\_REAL=arch-linux-real PETSC\_ARCH\_COMPLEX=arch-linux-complex}

\noindent To have the installer download and build \textbf{PETSc} and \textbf{SLEPc} local to \textbf{ncpaprop}, use approriate flags and options from the following:

\begin{description}
\item[\texttt{--with-localpetsc}]\hfill \\
\noindent
Download and build the current \texttt{maint} branch of \textbf{PETSc}.  Use of this flag also sets \texttt{--with-localslepc}.

\item[\texttt{--with-localslepc}]\hfill \\
\noindent
Download and build the latest version of \textbf{SLEPc}, which at the time of release is \textbf{3.12.2}.

\item[\texttt{--with-localpetscdebug}]\hfill \\
\noindent
Build the \texttt{debug} version of \textbf{PETSc} rather than the optimized version.

\item[\texttt{--disable-localpetscmpi}]\hfill \\
\noindent
Build \textbf{PETSc} without MPI support.  If this flag is not set, the installer will attempt to use a system-installed MPI library.

\item[\texttt{local\_petsc\_dir=VALUE}]\hfill \\
\noindent
Override the default \textbf{PETSc} installation location (defaults to \texttt{\$NCPAPROP\_ROOT/extern/petsc}).

\item[\texttt{local\_slepc\_dir=VALUE}]\hfill \\
\noindent
Override the default \textbf{SLEPc} installation location (defaults to \texttt{\$NCPAPROP\_ROOT/extern/slepc/slepc-\$VERSION)}.

\item[\texttt{local\_petsc\_arch\_real=VALUE}]\hfill \\
\noindent
Override the default \textbf{PETSc} real architecture name (defaults to \texttt{arch-\$OS-real} or \texttt{arch-\$OS-real}).

\item[\texttt{local\_petsc\_arch\_complex=VALUE}]\hfill \\
\noindent
Override the default \textbf{PETSc} complex architecture name (defaults to \texttt{arch-\$OS-complex} or \texttt{arch-\$OS-complex}).

\item[\texttt{local\_slepc\_version=VALUE}]\hfill \\
\noindent
Override the default \textbf{SLEPc} version number for determining the download filename.  This will generally be kept up-to-date by the \textbf{ncpaprop} maintainers, but is provided in the event of maintenance lag.
\end{description}

\noindent The installation process will check the system for the presence of BLAS libraries and MPI executables; whatever is not found will be automatically downloaded and installed locally by the \textbf{PETSc} installer.  This process is independent of the \texttt{--with-autodependencies} flag, which applies only to the direct prerequisites of \textbf{ncpaprop}.  If this is undesirable, \textbf{PETSc} and \textbf{SLEPc} should be installed manually.

\subsubsection{Compilation}

Once the configuration script has successfully run, the package can be built simply by running

\texttt{make}

\subsubsection{Testing}

\noindent TEST SCRIPTS ARE STILL UNDER REVISION AND MAY NOT PROVIDE ACCURATE RESULTS

To check proper functionality, a suite of test scripts have been written to compare the output of the simpler tools with that generated on the development machine.  This allows a sanity check to make sure that identical commands give identical results (within reasonable rounding errors).  To confirm that the compilation has been successful, run

\texttt{make test}

\noindent
and check the results.  If there are errors, check \texttt{test/testlog.txt} for clues as to why things didn't work properly.  A successful test should print the following output to the screen:

\begin{verbatim}
$ make test
make -C test test
make[1]: Entering directory `/home/claus/ncpaprop/v1.3/test'
rm testlog.txt

*** Running 2D Ray Theory Routines ***
/bin/bash run_raytrace.2d_test.bash >> ./testlog.txt

*** Running 3D Ray Theory Routines ***
/bin/bash run_raytrace.3d_test.bash >> ./testlog.txt

*** Running Effective-Sound-Speed Modal Routines ***
/bin/bash run_Modess_test.bash >> ./testlog.txt 

*** Running Complex Effective-Sound-Speed Modal Routines ***
/bin/bash run_CModess_test.bash >> ./testlog.txt 

*** Running One-Way Coupled Modal Routines ***
/bin/bash run_ModessRD1WCM_test.bash >> ./testlog.txt 

*** Running Wide-Angle Modal Routines ***
/bin/bash run_WMod_test.bash >> ./testlog.txt 

*** Checking 2D Ray Theory Calculation Results ***
/usr/bin/perl compare_raytrace.2d_test
raytrace.2d test OK

*** Checking 3D Ray Theory Calculation Results ***
/usr/bin/perl compare_raytrace.3d_test
raytrace.3d test OK

*** Checking Effective-Sound-Speed Modal Calculation Results ***
/bin/bash compare_Modess_test
Column 2 of Modess test OK
Column 3 of Modess test OK
Column 4 of Modess test OK

*** Checking Complex Effective-Sound-Speed Modal Calculation Results ***
/bin/bash compare_CModess_test
Column 2 of CModess test 1 OK
Column 3 of CModess test 1 OK
Column 4 of CModess test 1 OK
Column 5 of CModess test 1 OK

*** Checking One-Way Coupled Modal Calculation Results ***
/bin/bash compare_ModessRD1WCM_test
Column 2 of ModessRD1WCM test OK
Column 3 of ModessRD1WCM test OK
Column 4 of ModessRD1WCM test OK

*** Checking Wide-Angle Modal Calculation Results ***
/bin/bash compare_WMod_test
Column 2 of WMod test OK
Column 3 of WMod test OK
Column 4 of WMod test OK
make[1]: Leaving directory `/home/claus/ncpaprop/v1.3/test'
\end{verbatim}

\subsection{Typesetting this Manual}

This manual was produced with LaTeX. The required \verb+.tex+ and graphics files are archived in the \textbf{ncpaprop} package in the \verb+manual+ directory. The main \verb+.tex+ file, \verb+NCPA_prop_manual.tex+, is contained in the subdirectory \verb+ncpaprop_run_manual_TeX+ along with a utitlity script \verb+runtex.sh+. Entering \verb+. runtex.sh+ produces the manual as a pdf file and moves it to the \verb+manual+ directory. Entering \verb+. runtex.sh bib+ produces the bibliography using bibtex. 